%You can leave alone everything before Line 79.
\documentclass{article}
\usepackage{url,amsfonts, amsmath, amssymb, amsthm,color, enumerate}
 \usepackage{fullpage}
% Page layout
%\setlength{\textheight}{8.75in}
%\setlength{\columnsep}{2.0pc}
%\setlength{\textwidth}{6.5in}
%\setlength{\topmargin}{0in}
%\setlength{\headheight}{0.0in}
%\setlength{\headsep}{0.0in}
%\setlength{\oddsidemargin}{0in}
%\setlength{\evensidemargin}{0in}
%\setlength{\parindent}{1pc}
\newcommand{\shortbar}{\begin{center}\rule{5ex}{0.1pt}\end{center}}
%\renewcommand{\baselinestretch}{1.1}
% Macros for course info
\newcommand{\courseNumber}{EECS 545}
\newcommand{\courseTitle}{Machine Learning}
\newcommand{\semester}{Winter 2012}
% Theorem-like structures are numbered within SECTION units
\theoremstyle{plain}
\newtheorem{theorem}{Theorem}[section]
\newtheorem{lemma}[theorem]{Lemma}
\newtheorem{corollary}[theorem]{Corollary}
\newtheorem{proposition}[theorem]{Proposition}
\newtheorem{statement}[theorem]{Statement}
\newtheorem{conjecture}[theorem]{Conjecture}
\newtheorem{fact}{Fact}
%definition style
\theoremstyle{definition}
\newtheorem{definition}[theorem]{Definition}
\newtheorem{example}{Example}
\newtheorem{problem}[theorem]{Problem}
\newtheorem{exercise}{Exercise}
\newtheorem{algorithm}{Algorithm}
%remark style
\theoremstyle{remark}
\newtheorem{remark}[theorem]{Remark}
\newtheorem{reduction}[theorem]{Reduction}
%\newtheorem{question}[theorem]{Question}
\newtheorem{question}{Question}
%\newtheorem{claim}[theorem]{Claim}
%
% Proof-making commands and environments
\newcommand{\beginproof}{\medskip\noindent{\bf Proof.~}}
\newcommand{\beginproofof}[1]{\medskip\noindent{\bf Proof of #1.~}}
\newcommand{\finishproof}{\hspace{0.2ex}\rule{1ex}{1ex}}
\newenvironment{solution}[1]{\medskip\noindent{\bf Problem #1.~}}{\shortbar}

%====header======
\newcommand{\solutions}[4]{
%\renewcommand{\thetheorem}{{#2}.\arabic{theorem}}
\vspace{-2ex}
\begin{center}
{\small  \courseNumber, \courseTitle
\hfill {\Large \bf {#1} }\\
\semester, University of Michigan, Ann Arbor \hfill
{\em Date: #3}}\\
\vspace{-1ex}
\hrulefill\\
\vspace{4ex}
{\normalsize Project Progress Report}\\
%{\LARGE  TITLE #2}\\
\vspace{2ex}
\end{center}
\begin{trivlist}
\item \textsc{Team members:} {#4}
\end{trivlist}
\noindent
\shortbar
\vspace{3ex}
}
% math macros
\newcommand{\defeq}{\stackrel{\textrm{def}}{=}}
\newcommand{\Prob}{\textrm{Prob}}
%==
\usepackage{graphicx}
\begin{document}
%%%%%%%%%%%%%%%%%%%%%%%%%%%%%%%%%%%%%%%%%%%%%%%%%
%\solutions{Your name}{Problem Set Number}{Date of preparation}{Collaborators}{Prover}{Verifiers}
\solutions{}{}{\today}{\\ Keegan R. Kinkade, @kinkadek\\ Pedro d'Aquino, @pdaquino \\Shiva Ghose, @gshiva }
%%%%%%%%%%%%%%%%%%%%%%%%%%%%%%%%%%%%%%%%%%%%%%%%%
%\renewcommand{\theproblem}{\arabic{problem}} 
%%%%%%%%%%%%%%%%%%%%%%%%%%%%%%%%%%%%%%%%%%%%%%%%%
%
% Begin the solution for each problem by
% \begin{solution}{Problem Number} and ends it with \end{solution}
%
% the solution for Problem 

\begin{abstract}

In order for artificial intelligence agents to autonomously operate within a given environment, they must be capable of processing perceptual knowledge in such a manner as to intelligently interact with their environment. For a competitively driven AI agent, this becomes increasingly important, as the inability to intelligently process perceptual information will ultimately lead to defeat by those who are capable of such computation. To this end, the following describes the implementation of sophisticated machine learning techniques in an artificially driven tank battle simulator. Armed with such techniques, it is the goal of this project to provide a competitive, autonomous AI agent with the ability to determine how to best evade an opponent's attacks while striking in such a manner as to maximize the possibility of destroying the enemy. 

\end{abstract}

\section{Introduction}

Initially released in 2001 with the aim of assisting in learning object oriented programming, Robocode is an open source tank battling simulator implemented in Java. Each Robocode agent is designed with the ability to control a robotic tank, including differentially-driven movement, a $360\,^{\circ}\mathrm{}$ rotating turret, and an enemy scanning radar, in order to attempt to destroy other tanks with similar features but opposing strategies. Designers are capable of downloading other agent byte code in order to test their implementation, while the more committed are able to test their agents in Robocode tournaments. While the environment has made creating a simple working agent capable of moving, targeting, and shooting, perfecting agents to do well against multiple opponent strategies has proven to be difficult. Many of the top ranked agents incorporate sophisticated statistical analysis techniques in order to optimize their agents. In addition to such techniques, research has been done to optimize agents using both neural networks as well as genetic programming. However, little work has been done to incorporate sophisticated machine learning techniques designed to better inform Robocode agents when making decisions inside the battle environment.

\subsection*{Statement of the Problem}

The purpose of this project is to incorporate machine learning algorithms in a Robocode agent in order to optimize the agent's ability to stay alive during battles. Specifically, we wish to incorporate machine learning in two areas: targeting and evasion. With respect to evasion, we intend to gather environmental information on employing multiple missile evasion strategies against multiple opponents in an effort to determine which evasive strategy to employ in differing situations, ultimately reducing the likelihood of being struck by an opponent's missile.  Furthermore, we wish to use modeling machine learning algorithms in order to predict how an opponent will react upon learning that our agent has fired, thus allowing our agent to take advantage of the ability to predict opponent's behavior. Such an application of machine learning within multi-agent games provides the framework and motivation to explore different techniques in designing artificial agents capable of making perceptually informed decisions on how to best interact with their environment. 

\section{Proposed Approach}

Building a Robocode agent equipped with machine learning algorithms to better inform its decision making will be broken into two unique tasks: evasion and targeting. Each task will have a subset of strategies which the agent can choose to employ in order to best achieve a goal attached to the current task of evasion or targeting. In order to decide which strategy to employ for a given task, the agent will make use of machine learning techniques to analyze the current situation within the environment and choose the strategy which maximizes the likelihood of achieving the task goal. 

\subsection*{Evasion Strategy}

Within the task of evasion, there are two movement approaches that need to be considered:
\begin{itemize}

\item General movement 

\item Evasive movement 
\end{itemize}

Each approach to movement needs to be employed based on the situation,  and the agent will need to learn when which evasive movement strategy to employ when engaged in the general movement strategy.

\subsubsection*{General Movement }
In general the probability of getting hit by incoming fire is inversely proportional to how close the observer is to the firing point. We want our agent to move closer to the enemy when it has higher health to maximize the probability of hitting the enemy and stay further back when it is lower on health to better evade incoming fire. This can be modeled as series of attracting and repelling force interactions. Furthermore, the agent loses health each time it runs into walls in the environment. Thus, the walls will also be modeled as a set of repelling forces. For training purposes, the general movement of the agent will be to mirror the opponent at a fixed distance in order to avoid collisions with the enemy while better assessing the effectiveness of the evasion strategies. 

\subsubsection*{Evasive Movement }
Every time an agent fires a bullet, their energy drops proportionally to the speed of the bullet they fired. Built into Robocode is the ability to detect the energy drop of an opponent, and thus agents are capable of detecting when an enemy has fired a bullet. From such an energy drop, agents can determine the location a bullet was fired, and the velocity of the bullet, but not the direction with which the enemy fired in. Thus our agent will have to constantly monitor the actions taken by the opponent, and upon detecting the enemy has fired a bullet, choose an evasion strategy to employ. In order to do so, we will begin by creating a subset of relatively trivial evasion strategies and run the differing evasion strategies when fired upon in multiple training battles against multiple opponents. Each time an evasion strategy is employed, we will capture features in the environment as well as whether or not the strategy was effective in evading the bullet or not. We then will train support vector machines for each of the different evasion strategies using the features collected from the environment and a binary target variable corresponding to being hit by or evading the bullet. After training the SVMs, the agent will then choose which evasion strategy to employ when being fired upon by running the current environment features on each SVM, and choosing the strategy corresponding to the SVM which produces the largest margin for evading the bullet. 

\subsection*{Offensive Strategy}
The offensive strategy deals with modeling where the opponent will be if the agent were to fire at a given point of time under a given set of environmental conditions. A simple movement vector extrapolation will often fail against all but the most rudimentary opponents as they to employ advanced evasive strategies upon being fired at. Hence a more sophisticated prediction scheme is required to allow the agent to successfully target and attack the opponent. To this end, we plan on building up a probabilistic distribution of the opponents evasive reactions over time in order to better predict where to fire bullets. Furthermore, using training data in a similar fashion to that proposed in the evasion section, we will use an additional support vector machine to determine what situations lead to maximum likelihood of hitting the enemy when firing a bullet. This will ensure we only attempt to shoot at an enemy when we have a good chance of hitting them, thus reducing the amount of energy lost from making bad firing decisions. 
 
\section{Review of Related Work}
We have found some articles describing genetic programming approaches to Robocode, but none that combines it with reinforcement learning. Eisenstein, in 2003, was the first to use genetic programming in this context \cite{gp2}. He found that, while he was able to beat some hand coded adversaries, his robots had a hard time learning how to target, and were therefore more likely to try to ram their opponents.\\

In \cite{gp1}, the authors build upon Eisenstein's work and use genetic programming to evolve tank strategies for a robot in the HaikuBot category (which allows robots whose code is no longer than 4 lines). They evolved a population of 256 robots over approximately 400 generations. Their robot was ranked 3rd in the HaikuBot category.\\

In one of the very few academic mentions of Robocode outside of the artificial intelligence field, Kobayashi et al. describe a targeting strategy that was only mildly successful \cite{strategies}

\section{Experimental Results}

\subsection*{Evaluation Methodology}
In order to measure the agent's performance, we require a performance measure. The agent's health is a good place to start, though that alone will have some drawbacks.\\

In Robocode, in order to fire, an agent must use some of its own health. If the agent's health were it only performance measure, it is foreseeable that an agent will quickly get stuck in a local minimum by choosing not to fire at the enemy at all and instead try to survive by simply dodging the enemy's incoming fire. While this might work initially, as the agent will be drawn closer and closer to the enemy as its opponent loses health, it is easy to imagine that the agent will still get considerable damage.\\

Hence we need to devise a reward scheme that rewards the agent for effectively attacking the enemy, effectively dodging incoming fire and staying alive. These rewards will probably be set manually, although if time permits, these rewards can also be optimized using machine learning techniques.

\section{Future Milestones}

\section{Conclusion}

\bibliographystyle{IEEEtranS}
\bibliography{sources.bib}

\end{document}


\def\therefore{\boldsymbol{\text{ }
\leavevmode
\lower0.4ex\hbox{$\cdot$}
\kern-.5em\raise0.7ex\hbox{$\cdot$}
\kern-0.55em\lower0.4ex\hbox{$\cdot$}
\thinspace\text{ }}}
