%You can leave alone everything before Line 79.
\documentclass{article}
\usepackage{url,amsfonts, amsmath, amssymb, amsthm,color, enumerate}
% Page layout
\setlength{\textheight}{8.75in}
\setlength{\columnsep}{2.0pc}
\setlength{\textwidth}{6.5in}
\setlength{\topmargin}{0in}
\setlength{\headheight}{0.0in}
\setlength{\headsep}{0.0in}
\setlength{\oddsidemargin}{0in}
\setlength{\evensidemargin}{0in}
\setlength{\parindent}{1pc}
\newcommand{\shortbar}{\begin{center}\rule{5ex}{0.1pt}\end{center}}
%\renewcommand{\baselinestretch}{1.1}
% Macros for course info
\newcommand{\courseNumber}{EECS 545}
\newcommand{\courseTitle}{Machine Learning}
\newcommand{\semester}{Winter 2012}
% Theorem-like structures are numbered within SECTION units
\theoremstyle{plain}
\newtheorem{theorem}{Theorem}[section]
\newtheorem{lemma}[theorem]{Lemma}
\newtheorem{corollary}[theorem]{Corollary}
\newtheorem{proposition}[theorem]{Proposition}
\newtheorem{statement}[theorem]{Statement}
\newtheorem{conjecture}[theorem]{Conjecture}
\newtheorem{fact}{Fact}
%definition style
\theoremstyle{definition}
\newtheorem{definition}[theorem]{Definition}
\newtheorem{example}{Example}
\newtheorem{problem}[theorem]{Problem}
\newtheorem{exercise}{Exercise}
\newtheorem{algorithm}{Algorithm}
%remark style
\theoremstyle{remark}
\newtheorem{remark}[theorem]{Remark}
\newtheorem{reduction}[theorem]{Reduction}
%\newtheorem{question}[theorem]{Question}
\newtheorem{question}{Question}
%\newtheorem{claim}[theorem]{Claim}
%
% Proof-making commands and environments
\newcommand{\beginproof}{\medskip\noindent{\bf Proof.~}}
\newcommand{\beginproofof}[1]{\medskip\noindent{\bf Proof of #1.~}}
\newcommand{\finishproof}{\hspace{0.2ex}\rule{1ex}{1ex}}
\newenvironment{solution}[1]{\medskip\noindent{\bf Problem #1.~}}{\shortbar}

%====header======
\newcommand{\solutions}[4]{
%\renewcommand{\thetheorem}{{#2}.\arabic{theorem}}
\vspace{-2ex}
\begin{center}
{\small  \courseNumber, \courseTitle
\hfill {\Large \bf {#1} }\\
\semester, University of Michigan, Ann Arbor \hfill
{\em Date: #3}}\\
\vspace{-1ex}
\hrulefill\\
\vspace{4ex}
{\LARGE Project Proposal #2}\\
\vspace{2ex}
\end{center}
\begin{trivlist}
\item \textsc{Team members:} {#4}
\end{trivlist}
\noindent
\shortbar
\vspace{3ex}
}
% math macros
\newcommand{\defeq}{\stackrel{\textrm{def}}{=}}
\newcommand{\Prob}{\textrm{Prob}}
%==
\usepackage{graphicx}
\begin{document}
%%%%%%%%%%%%%%%%%%%%%%%%%%%%%%%%%%%%%%%%%%%%%%%%%
%\solutions{Your name}{Problem Set Number}{Date of preparation}{Collaborators}{Prover}{Verifiers}
\solutions{}{}{\today}{\\ Keegan R. Kinkade, @kinkadek\\ Pedro d'Aquino, @pdaquino \\Shiva Ghose, @gshiva }
%%%%%%%%%%%%%%%%%%%%%%%%%%%%%%%%%%%%%%%%%%%%%%%%%
%\renewcommand{\theproblem}{\arabic{problem}} 
%%%%%%%%%%%%%%%%%%%%%%%%%%%%%%%%%%%%%%%%%%%%%%%%%
%
% Begin the solution for each problem by
% \begin{solution}{Problem Number} and ends it with \end{solution}
%
% the solution for Problem 

\section*{Introduction}

Initially released in 2001 with the aim of assisting in learning object oriented programming, Robocode is an open source tank battling simulator implemented in Java. Each Robocode agent is designed with the ability to control a robotic tank, including differentially-driven movement, a $360\,^{\circ}\mathrm{}$ rotating turret, and an enemy scanning radar, in order to attempt to destroy other tanks with similar features but opposing strategies. Designers are capable of downloading other agent byte code in order to test their implementation, and their also exists leagues in which Robocode tournaments are held. While the environment has made creating a simple working agent capable of moving, targeting, and shooting, perfecting agents to do well against multiple opponent strategies has proven to be difficult. Many of the top ranked agents incorporate sophisticated statistical analysis techniques in order to optimize their agents. In addition to such techniques, research has been done to optimize agents using both neural networks as well as genetic programming. 

\section*{Statement of the Problem}

The purpose of this project is to incorporate machine learning algorithms in a Robocode agent in order to optimize the agent's ability to stay alive during battles. Specifically, we wish to incorporate machine learning in two areas: targeting and evasion. With respect to evasion, we intend on using reinforced learning in combination with genetic algorithms to optimize multiple missile evasion strategies and employ them in such a manner as to reduce the likelihood of being struck by an opponent's missile. Furthermore, we wish to use modeling machine learning algorithms in order to predict how an opponent will react upon learning that our agent has fired, thus allowing our agent to take advantage of the ability to predict opponent's behavior. 

\section*{Proposed Approach}

\section*{Evaluation Methodology}

\section*{Review of Related Work}
We have found some articles describing genetic programming
approaches to Robocode, but none that combines it with reinforcement
learning. Eisenstein, in 2003, was the first to use genetic programming
in this context \cite{gp2}. He found that,
while he was able to beat some hand-coded adversaries, his robots
had a hard time learning how to target, and were therefore more
likely to try to ram their opponents.

In \cite{gp1}, the authors build upon Eisenstein's work and use genetic programming to evolve tank
strategies for a robot in the HaikuBot category (which allows
robots whose code is no longer than 4 lines). They evolved a population
of 256 robots over approximately 400 generations. Their robot was ranked
3rd in the HaikuBot category.

In one of the very few academic mentions of Robocode outside of the artificial
intelligence field, Kobayashi et al. describe a targeting strategy that was
only mildly successful \cite{strategies}
\bibliographystyle{IEEEtranS}
\bibliography{sources}

\end{document}

\def\therefore{\boldsymbol{\text{ }
\leavevmode
\lower0.4ex\hbox{$\cdot$}
\kern-.5em\raise0.7ex\hbox{$\cdot$}
\kern-0.55em\lower0.4ex\hbox{$\cdot$}
\thinspace\text{ }}}
